\documentclass[a4paper,10pt]{article}
\usepackage{geometry}
\geometry{margin=1in}
\usepackage{enumitem}
\usepackage{hyperref}
\usepackage[normalem]{ulem} % Avoid changes to default LaTeX behavior
% \usepackage{helvet}
% \renewcommand{\familydefault}{\sfdefault}
\pagenumbering{gobble} % No page numbers

\geometry{margin=0.7in}

\setlength{\parskip}{0.5pt}
\setlist{nosep}  
\usepackage[compact]{titlesec}
\usepackage{setspace}
\setstretch{1.0}

\begin{document}

\begin{center}
    \huge \textbf{YONGKANG CHENG}\\
    \vspace{0.2cm}
    \small
    \href{https://chengyongkang.me/}{\uline{chengyongkang.me}} \,\textbar{}\,
    437-663-2855 \,\textbar{}\,
    \href{https://www.github.com/Ken-2511/}{\uline{github.com/Ken-2511}} \,\textbar{}\,
    \href{mailto:iwmain@outlook.com}{\uline{iwmain@outlook.com}} \,\textbar{}\,
    \href{https://www.linkedin.com/in/chengyongkang/}{\uline{linkedin.com/in/chengyongkang}}
\end{center}

\vskip -2ex
\noindent\rule{\linewidth}{2pt}

\section*{\textbf{EDUCATION}}
\textbf{University of Toronto (St. George Campus), Toronto, ON} \hfill Sep 2023 - May 2028 (expected)\\
Bachelor of Applied Science in Computer Engineering + PEY Co-op \\
\textit{GPA: 3.92/4.0 (Top 30 among first-year ECE students)} \\
Relevant Courses: Applied Fundamentals of Deep Learning, Software Design and Communication

\noindent\rule{\linewidth}{1pt}

\section*{\textbf{TECHNICAL SKILLS}}
\begin{itemize}[leftmargin=0.2in]
    \item \textbf{Programming:} Python, C/C++, JavaScript/TypeScript, Bash
    \item \textbf{Frameworks:} PyTorch, FastAPI, React
    \item \textbf{Tools:} Linux, Docker, Kubernetes, Terraform, Helm, Git, Datadog, AWS
    \item \textbf{Cloud and Infrastructure:} AWS VPC, Nginx, SSH, CI/CD pipelines
\end{itemize}

\noindent\rule{\linewidth}{1pt}

\section*{\textbf{EXPERIENCE}}

\noindent\href{https://github.com/Ken-2511/HandwritingRecognition}{\uline{
\textbf{Infrastructure Intern, Handwritten Text Recognition}}} \hfill Jun 2024 - Aug 2024
\begin{itemize}[leftmargin=0.2in]
    \item Deployed CRNN models on cloud infrastructure, leveraging Docker containers and Kubernetes orchestration.
    \item Reduced deployment times by 35\% by optimizing CI/CD pipelines with GitLab.
    \item Improved monitoring and error detection by integrating Datadog and Opsgenie into workflows.
\end{itemize}

\vspace{0.3cm}
\noindent\href{https://github.com/Ken-2511/WillPower}{\uline{
\textbf{Infrastructure Engineer, WillPower Monitoring System}}} \hfill Jan 2025 - Present
\begin{itemize}[leftmargin=0.2in]
    \item Built a modular infrastructure system using Docker, FastAPI, and Nginx for time management monitoring.
    \item Automated deployment of services using Terraform, cutting manual provisioning time by 50\%.
    \item Currently exploring cost-saving measures in AWS through resource usage analysis.
\end{itemize}

\vspace{0.3cm}
\noindent\href{https://github.com/VolunTrack/Web}{\uline{
\textbf{Frontend Manager, Voluntrack.org (Non-Profit)}}} \hfill May 2024 - Present
\begin{itemize}[leftmargin=0.2in]
    \item Coordinated a 4-person frontend team using React.js to renew the web interface.
    \item Enhanced API request handling and error monitoring, improving performance by 20\%.
\end{itemize}

\vspace{0.3cm}
\noindent\href{https://chengyongkang.me/chat}{\uline{
\textbf{Self-Clone Chatbot with Diary Database}}} \hfill Oct 2024 - Present
\begin{itemize}[leftmargin=0.2in]
    \item Built a self-hosted AI-powered chatbot that replicates personal interaction styles, deployed using React.js, FastAPI, and Nginx on a Raspberry Pi.
    \item Integrated OpenAI API and a NoSQL database for real-time Q\&A functionality with personal diary data.
    \item Ensured secure and seamless remote access by implementing TLS encryption, DDNS, and optimizing for daily traffic from personal networks.
\end{itemize}

\noindent\rule{\linewidth}{1pt}

\section*{\textbf{PROJECTS}}

\noindent\href{https://github.com/Ken-2511/ECE241-Project}{\uline{
\textbf{Verilog Pac-Man Game (University of Toronto)}}} \hfill Nov 2024
\begin{itemize}[leftmargin=0.2in]
    \item Created an FPGA-based Pac-Man game using Verilog, supporting VGA output and PS/2 input.
    \item Enhanced modularity in game design by automating Verilog module testing pipelines.
\end{itemize}

\vspace{0.3cm}
\noindent\href{https://github.com/Ken-2511/RainBirthdayGift}{\uline{
\textbf{RainBirthdayGift – AI-Driven Chatbot}}} \hfill April 2024
\begin{itemize}[leftmargin=0.2in]
    \item Built a lightweight chatbot using C\#, .NET, and WPF with OpenAI API integration.
    \item Reduced API costs by implementing asynchronous calls and efficient logging.
\end{itemize}

\noindent\rule{\linewidth}{1pt}

\section*{\textbf{AWARDS \& ACCOMPLISHMENTS}}

\noindent\textbf{ECE Awards \& Dean's List Scholar (UofT)} \hfill Sep 2024
\begin{itemize}[leftmargin=0.2in]
    \item Recognized for outstanding academic performance (GPA 3.92/4.0).
\end{itemize}

\vspace{0.3cm}
\noindent\textbf{American Computer Science League (ACSL) - Bronze Prize} \hfill Jan 2021
\begin{itemize}[leftmargin=0.2in]
    \item Placed in top 10\% overall, with top-20\% scores in the 4th round, after 60 hours of training.
\end{itemize}

\end{document}
