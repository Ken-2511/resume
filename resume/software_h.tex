\documentclass[letterpaper,10pt]{article}
\usepackage{geometry}
\geometry{margin=0.65in}
\usepackage{enumitem}
\usepackage[hidelinks]{hyperref}
\usepackage[normalem]{ulem}
\pagenumbering{gobble}

\setlength{\parskip}{0.5pt}
\setlist{nosep,leftmargin=0.18in}
\usepackage[compact]{titlesec}
\titlespacing*{\section}{0pt}{4pt}{2pt}
\usepackage{setspace}
\setstretch{1.0}

\begin{document}

%------------------------------------------
% HEADER
%------------------------------------------
\begin{center}
    {\huge \textbf{Yongkang Cheng}}\\[2pt]
    {\small
    \href{https://chengyongkang.me/}{\uline{chengyongkang.me}} \;\textbar\; 
    437-663-2855 \;\textbar\; 
    \href{https://www.github.com/Ken-2511/}{\uline{github.com/Ken-2511}} \;\textbar\; 
    \href{mailto:iwmain@outlook.com}{\uline{iwmain@outlook.com}} \;\textbar\; 
    \href{https://www.linkedin.com/in/chengyongkang/}{\uline{linkedin.com/in/chengyongkang}}}
\end{center}
\vskip -1.7ex
\noindent\rule{\linewidth}{1.6pt}

%------------------------------------------
% EDUCATION
%------------------------------------------
\section*{\textbf{EDUCATION}}
\textbf{University of Toronto (St. George), Toronto, ON} \hfill Sep 2023 -- May 2028 (expected)\\
BASc in Computer Engineering + PEY Co-op (cGPA: 3.87/4.0, Score: 88.6/100) \\
Relevant Courses: Applied Deep Learning Fundamentals (PyTorch), Software Design \& Communication (C++), Computer Architecture, Operating Systems (ongoing), Algorithms \& Data Structures (ongoing)

\noindent\rule{\linewidth}{1pt}

%------------------------------------------
% EXPERIENCE
%------------------------------------------
\section*{\textbf{EXPERIENCES}}

\noindent\textbf{Research Assistant, Spiking Neural Network Edge Device Deployment} \hfill Sep 2025 - Present\\
\textit{Research Intern, X-Lab, University of Toronto} \hfill \textit{Toronto, ON}
\begin{itemize}[leftmargin=0.2in]
	\item Took over Verilog implementation of an SNN from a teammate; verified modules with ModelSim testbenches.
	\item Quantized the CNN layers to 8-bit using TorchAO, working on post-training quantization on SNN layers, accelerating mixed-precision inference using HaiLo-8 on Raspberry Pi for real-time low-power inference with small precision drop.
\end{itemize}

\vspace{0.2em}
\noindent
\textbf{Research Assistant, Ultra-Wideband Receiver Design} \hfill May 2025 - Jul 2025\\
\textit{Research Intern, X-Lab, University of Toronto} \hfill \textit{Toronto, ON}
\begin{itemize}[leftmargin=0.2in]
    \item Collaborated in a 2-person team, verified a hybrid 4-PPM + 8-PSK TX chip pre-tapeout; built Python/Simulink pipelines for 2 ns symbol sync and carrier recovery under discontinuous 4.6 GHz.
    \item Achieved error-free demodulation across 2,500 symbols under $\geq 16\,\mathrm{dB}$ SNR (AWGN) and $\pi/16$ phase jitter.
    \item Presented at Undergraduate Engineering Research Day with a \href{https://docs.google.com/presentation/d/1h4lmc_HQLzNvtGE4oE1jedOlNVZjL3iztEyGKnC-ico/edit?usp=sharing}{\uline{poster}} and an interactive \href{https://github.com/Ken-2511/ppm-psk-visualize}{\uline{demo}}, engaged 50+ attendees.
\end{itemize}

\noindent\rule{\linewidth}{1pt}

%------------------------------------------
% PROJECTS
%------------------------------------------
\section*{\textbf{PROJECTS}}
\noindent\href{https://github.com/Ken-2511/HandwritingRecognition}{\uline{\textbf{Handwritten Text Recognition (CRNN)}}} \hfill Jun 2024 -- Aug 2024\\
\textit{Course Project \textbar{} PyTorch, OpenCV, TensorBoard}
\begin{itemize}
    \item Led a team of 4 to develop a PyTorch-based CRNN (ResNet-50 + BiLSTM) for handwritten text recognition on a RTX-4090, achieving 87\% word- and 95\% char-level accuracy on 10k+ samples.
    \item Augmented IAM and CVL dataset with random distortions, and generated synthetic data using EMNIST characters.
    \item Implemented connected-component word segmentation; 1024\,x\,1024 inference in \textless 4s on CPU.
\end{itemize}

\vspace{0.2em}
\noindent\href{https://github.com/Ken-2511/city-mapify}{\uline{\textbf{City Mapify – Interactive Mapping Engine}}} \hfill Jan 2025 -- Apr 2025\\
\textit{Course Project \textbar{} C++, OpenStreetMap, GTK}
\begin{itemize}
	\item Built C++ mapping engine parsing 2GB OpenStreetMap data with QuadTree indexing at 60 FPS.
	\item Implemented A*/Dijkstra pathfinding and delivery optimization (Simulated Annealing, ACO) for 250+ packages.
\end{itemize}

\vspace{0.2em}
\noindent\href{https://chengyongkang.me/chat}{\uline{\textbf{Self-Hosted Chatbot with Diary DB}}} \hfill Oct 2024\\
\textit{Personal Project \textbar{} React, FastAPI, DDNS, Nginx, NoSQL}
\begin{itemize}
	\item Designed a journaling software with OpenAI API, generating insightful feedbacks for over 750 diary entries.
    \item Deployed full-stack chatbot replicating personal style; configured TLS, DDNS, and reverse proxy for secure access.
    \item Integrated OpenAI API and vector search for context-aware Q\&A over personal diaries.
\end{itemize}

\noindent\rule{\linewidth}{1pt}

%------------------------------------------
% SKILLS
%------------------------------------------
\section*{\textbf{SKILLS}}
\begin{itemize}
    \item \textbf{Languages:} Python, C/C++, Node.js, Java, Verilog, Assembly, MATLAB/Simulink
    \item \textbf{Web/Backend:} React, FastAPI, Flask, Docker, Nginx, SQL/NoSQL
    \item \textbf{ML/Data:} PyTorch, NumPy, OpenCV, Pandas, LangChain, MCP
    \item \textbf{Tools:} Linux, Git, SSH, Firebase, Raspberry Pi, 3D Printing, STM32, Altium Designer
\end{itemize}

\noindent\rule{\linewidth}{1pt}

%------------------------------------------
% AWARDS
%------------------------------------------
\section*{\textbf{AWARDS}}

\begin{itemize}[leftmargin=0.2in]
	\item \textbf{University of Toronto Excellence Award (UTEA)} \hfill Apr 2025\\
    \$7,500 scholarship for 6 students among 2nd to 4th ECE for summer research in X-Lab.
	\item \textbf{ECE Awards} \hfill Sep 2024\\
	Awarded to top 30 students in the first-year ECE program out of 300+ students.
\end{itemize}

\end{document}