\documentclass[letterpaper,10pt]{article}
\usepackage{geometry}
\geometry{margin=0.65in}
\usepackage{enumitem}
\usepackage[hidelinks]{hyperref}
\usepackage[normalem]{ulem}
\pagenumbering{gobble}

\setlength{\parskip}{0.5pt}
\setlist{nosep,leftmargin=0.18in}
\usepackage[compact]{titlesec}
\titlespacing*{\section}{0pt}{4pt}{2pt}
\usepackage{setspace}
\setstretch{1.0}

\begin{document}

%------------------------------------------
% HEADER
%------------------------------------------
\begin{center}
    {\huge \textbf{Yongkang Cheng}}\\[2pt]
    \vskip 1ex
    {\small
    \href{https://chengyongkang.me/}{\uline{chengyongkang.me}} \;\textbar\; 
    437-663-2855 \;\textbar\;
    \href{https://www.github.com/Ken-2511/}{\uline{github.com/Ken-2511}} \;\textbar\; 
    \href{mailto:iwmain@outlook.com}{\uline{iwmain@outlook.com}} \;\textbar\; 
    \href{https://www.linkedin.com/in/chengyongkang/}{\uline{linkedin.com/in/chengyongkang}}}
\end{center}
\vskip -1.7ex
\noindent\rule{\linewidth}{1.6pt}

%------------------------------------------
% EDUCATION
%------------------------------------------
\section*{\textbf{EDUCATION}}
\textbf{University of Toronto (St. George), Toronto, ON} \hfill Sep 2023 -- May 2028 (expected)\\
BASc in Computer Engineering + PEY Co-op, cGPA: 3.87/4.0 (88.6\%)\\
Relevant Courses: Digital Electronics (in progress), Operating Systems (in progress), Algorithms and Data Structures(in progress), Digital Systems (Verilog), Computer Organization (Assembly), Software Design and Communication (C++)

\noindent\rule{\linewidth}{1pt}

%------------------------------------------
% EXPERIENCES
%------------------------------------------
\section*{\textbf{EXPERIENCES}}

\noindent\textbf{Research Assistant, Wireless Power Transfer Coil Design} \hfill Jul 2025 - Aug 2025\\
\textit{Research Intern, X-Lab, University of Toronto} \hfill \textit{Toronto, ON}
\begin{itemize}[leftmargin=0.2in]
    \item Designed a resonant 13.56 MHz coil pair for brain–computer interface power delivery across 20 mm separation.
    \item Optimized coupling coefficient (k$>$0.01) and quality factor (Q$>$28) through HFSS sweeps and EM simulations.
    \item Created PCBs with tuning networks; fabricated and tested 3 TX and 11 RX prototypes to validate efficiency.
    \item Tuned resonance under load and analyzed link performance to improve stability of wireless power transfer.
\end{itemize}

\vspace{0.2cm}
\noindent
\textbf{Research Assistant, Ultra-Wideband Receiver Design} \hfill May 2025 - Jul 2025\\
\textit{Research Intern, X-Lab, University of Toronto} \hfill \textit{Toronto, ON}
\begin{itemize}[leftmargin=0.2in]
    \item Verified a hybrid 2-PPM + 8-PSK TX chip pre-tapeout; built Python/Simulink pipelines for 2 ns symbol sync and carrier recovery under discontinuous 4 GHz.
	\item Built pulse-position detection and K-means phase calibration for PPM demodulation and constellation adjustment.
    \item Achieved error-free demodulation across 2,500 symbols under $\geq 13\,\mathrm{dB}$ SNR (AWGN) and $\pi/16$ phase jitter.
    \item Automated simulation/test flows with Python scripts for waveform analysis and regression testing.
    \item Presented at Undergraduate Engineering Research Day with a \href{https://docs.google.com/presentation/d/1h4lmc_HQLzNvtGE4oE1jedOlNVZjL3iztEyGKnC-ico/edit?usp=sharing}{\uline{poster}} and an interactive demo \href{https://github.com/Ken-2511/ppm-psk-visualize}{\uline{site}}.
\end{itemize}

\noindent\rule{\linewidth}{1pt}

%------------------------------------------
% PROJECTS
%------------------------------------------
\section*{\textbf{PROJECTS}}

\noindent\href{https://github.com/alexzjm/ece243-sound-synthesizer}{\uline{
\textbf{FPGA Polyphonic Synthesizer}}} \hfill Mar 2025
\begin{itemize}[leftmargin=0.2in]
  \item Built a 20-voice polyphonic synthesizer in C on a Nios-V RISC-V soft-core, streaming 8 kHz audio via on-chip FIFO.
  \item Used Q15 fixed-point arithmetic, LUTs, and shift-based division to replace floating-point, enabling real-time DSP mixing and ADSR envelopes.
  \item Implemented interrupt-driven PS/2 keyboard input with a double-buffered 320×240 VGA interface.
  \item Developed simple graphics primitives and interrupt-driven APIs for responsive, low-latency user interaction.
\end{itemize}


\vspace{0.2cm}
\noindent\href{https://github.com/Ken-2511/ECE241-Project}{\uline{
\textbf{Verilog Pac-Man Game}}} \hfill Nov 2024
\begin{itemize}[leftmargin=0.2in]
    \item Built (with one collaborator) an FPGA Pac‑Man in Verilog with PS/2 input and VGA output on DE1-SoC.
    \item Implemented hierarchical FSM game logic and a custom VGA controller with cycle‑accurate 320×240 scan timing.
    \item Resolved signal synchronization and state logic issues using ModelSim; automated asset conversion via OpenCV.
\end{itemize}

\noindent\rule{\linewidth}{1pt}

%------------------------------------------
% SKILLS
%------------------------------------------
\section*{\textbf{SKILLS}}
\begin{itemize}[leftmargin=0.2in]
    \item \textbf{Programming:} Python, C/C++, Verilog, Assembly (RISC-V), MATLAB/Simulink
    \item \textbf{EDA Tools:} Quartus, ModelSim, LTSpice, Ansys HFSS, Altium Designer, SUE, MAX
    \item \textbf{Hardware Tools:} PCB Design, Oscilloscope, Soldering, 3D Printing, Vector Network Analyzer
    \item \textbf{3D Modeling:} Rhinoceros, Blender, 3Ds Max
\end{itemize}

\noindent\rule{\linewidth}{1pt}

%------------------------------------------
% AWARDS & ACCOMPLISHMENTS
%------------------------------------------
\section*{\textbf{AWARDS}}

\begin{itemize}[leftmargin=0.2in]
	\item \textbf{University of Toronto Excellence Award (UTEA)} \hfill Apr 2025\\
    \$7,500 scholarship for 6 students among 2nd to 4th ECE for research potential.
	\item \textbf{ECE Awards} \hfill Sep 2024\\
	Awarded to top 30 students in the first-year ECE program out of 300+ students.
\end{itemize}

\end{document}