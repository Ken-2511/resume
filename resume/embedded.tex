\documentclass[letterpaper,10pt]{article}
\usepackage{geometry}
\geometry{margin=0.65in}
\usepackage{enumitem}
\usepackage[hidelinks]{hyperref}
\usepackage[normalem]{ulem}
\pagenumbering{gobble}

\setlength{\parskip}{0.5pt}
\setlist{nosep,leftmargin=0.18in}
\usepackage[compact]{titlesec}
\titlespacing*{\section}{0pt}{4pt}{2pt}
\usepackage{setspace}
\setstretch{1.0}

\begin{document}

%------------------------------------------
% HEADER
%------------------------------------------
\begin{center}
    {\huge \textbf{Yongkang Cheng}}\\[2pt]
    \vskip 1ex
    {\small
    \href{https://chengyongkang.me/}{\uline{chengyongkang.me}} \;\textbar\; 
    437-663-2855 \;\textbar\;
    \href{https://www.github.com/Ken-2511/}{\uline{github.com/Ken-2511}} \;\textbar\; 
    \href{mailto:iwmain@outlook.com}{\uline{iwmain@outlook.com}} \;\textbar\; 
    \href{https://www.linkedin.com/in/chengyongkang/}{\uline{linkedin.com/in/chengyongkang}}}
\end{center}
\vskip -1.7ex
\noindent\rule{\linewidth}{1.6pt}

%------------------------------------------
% EDUCATION
%------------------------------------------
\section*{\textbf{EDUCATION}}
\textbf{University of Toronto}, Toronto, ON \hfill Sep 2023 -- May 2028 (expected)\\
BASc Computer Engineering + PEY Co-op, cGPA: 3.87/4.0 (88.6\%)\\
Relevant: Digital Systems, Computer Organization, (In progress) Digital Electronics, Operating Systems

\noindent\rule{\linewidth}{1pt}

%------------------------------------------
% EXPERIENCE
%------------------------------------------
\section*{\textbf{EXPERIENCES}}

\noindent\textbf{Research Assistant, Spiking Neural Network FPGA Deployment} \hfill Sep 2025 - Present\\
\textit{Research Intern, X-Lab, University of Toronto} \hfill \textit{Toronto, ON}
\begin{itemize}[leftmargin=0.2in]
	\item Took over Verilog implementation of an SNN from a teammate; verified modules with ModelSim testbenches.
	\item Prototyping FPGA deployment flow with Vivado, validated initial design on Nexys board, preparing SNN integration.
\end{itemize}

\vspace{0.3em}
\noindent\textbf{Research Assistant, Ultra-Wideband Receiver Design} \hfill May 2025 - Jul 2025\\
\textit{Research Intern, X-Lab, University of Toronto} \hfill \textit{Toronto, ON}
\begin{itemize}[leftmargin=0.2in]
    \item Collaborated in a 2-person team, verified a hybrid 4-PPM + 8-PSK TX chip pre-tapeout; built Python/Simulink pipelines for 2 ns symbol sync and carrier recovery under discontinuous 4.6 GHz.
	\item Built pulse-position detection and K-means phase calibration for constellation stabilization and PPM demodulation.
    \item Achieved error-free demodulation across 2,500 symbols under $\geq 16\,\mathrm{dB}$ SNR (AWGN) and $\pi/16$ phase jitter.
    \item Presented at Undergraduate Engineering Research Day with a \href{https://docs.google.com/presentation/d/1h4lmc_HQLzNvtGE4oE1jedOlNVZjL3iztEyGKnC-ico/edit?usp=sharing}{\uline{poster}} and an interactive \href{https://github.com/Ken-2511/ppm-psk-visualize}{\uline{demo}}, engaged 50+ attendees.
\end{itemize}

\noindent\rule{\linewidth}{1pt}

%------------------------------------------
% PROJECTS
%------------------------------------------
\section*{\textbf{PROJECTS}}
\noindent\href{https://github.com/alexzjm/ece243-sound-synthesizer}{\uline{
    \textbf{FPGA Polyphonic Synthesizer}}} \hfill Mar 2025
\begin{itemize}[leftmargin=0.2in]
  \item Developed a 20-voice soft-core audio engine (C on Nios-V) streaming 8 kHz Q15 samples via audio FIFO; optimized kernels by replacing floating-point with fixed-point arithmetic.
  \item Designed and integrated interrupt service routines for PS/2 input and a double-buffered 320×240 VGA display pipeline, ensuring synchronized graphics rendering with deterministic latency.
  \item Developed simple graphics primitives and interrupt-driven APIs for responsive, low-latency user interaction.
\end{itemize}

\vspace{0.3em}
\noindent\href{https://github.com/Ken-2511/ECE241-Project}{\uline{
\textbf{Verilog Pac-Man Game (DE1-SoC)}}} \hfill Nov 2024
\begin{itemize}[leftmargin=0.2in]
    \item Created a Pac-Man-style FPGA game using Verilog on DE1-SoC supporting PS/2 keyboard input and VGA output.
    \item Debugged signal sync and FSM logic issues using ModelSim, automated the debug process with customized testbenches and .do files, ensured smooth and bug-free gameplay.
    \item Prototyped the game using Pygame for agile development, automated image conversion using Python + OpenCV.
\end{itemize}

\vspace{0.3em}
\noindent\textbf{City Mapify (C++ Performance Engine)} \hfill Jan 2025 -- Apr 2025
\begin{itemize}[leftmargin=0.2in]
  \item Implemented QuadTree spatial index and pathfinding (Dijkstra/A*/metaheuristics) with cache-friendly adjacency layout enabling 60 FPS rendering on 2GB map data.
  \item Designed modular pathfinding kernel with templated weight/heuristic for fast extensibility.
\end{itemize}

\noindent\rule{\linewidth}{1pt}

%------------------------------------------
% TECHNICAL SKILLS
%------------------------------------------

\section*{\textbf{SKILLS}}
\begin{itemize}[leftmargin=0.2in]
    \item \textbf{Programming:} C/C++, Python, Verilog, Assembly (RISC-V), MATLAB/Simulink, JavaScript, Java
    \item \textbf{Systems:} Linux system programming, multithreading, process control, memory management, device drivers
    \item \textbf{EDA Tools:} Quartus, Vivado, ModelSim, LTSpice, Ansys HFSS, Altium Designer, Rhinoceros, Blender, 3Ds Max
    \item \textbf{Tools \& Libraries:} I2C, SPI, OpenCV, PyTorch, Pandas, FastAPI, Nginx, Git, Linux
    \item \textbf{Hardware:} Raspberry Pi (5, zero, pico), STM32, Soldering, 3D Printing, Oscilloscope, Logic Analyzer, Multimeter
\end{itemize}

\noindent\rule{\linewidth}{1pt}

%------------------------------------------
% AWARDS
%------------------------------------------
\section*{\textbf{AWARDS}}
\begin{itemize}[leftmargin=0.2in]
	\item \textbf{University of Toronto Excellence Award (UTEA)} \hfill Apr 2025\\
    \$7,500 scholarship for 6 students among 2nd to 4th ECE for research potential.
	\item \textbf{ECE Awards} \hfill Sep 2024\\
	Awarded to top 30 students in the first-year ECE program out of 300+ students.
\end{itemize}

\end{document}
