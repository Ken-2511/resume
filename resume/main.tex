\documentclass[letterpaper,10pt]{article}
\usepackage{geometry}
\geometry{margin=1in}
\usepackage{enumitem}
\usepackage{hyperref}
\usepackage[normalem]{ulem} % 避免 ulem 修改默认的 LaTeX 行为
% \usepackage{helvet}
% \renewcommand{\familydefault}{\sfdefault}
\pagenumbering{gobble} % 取消页码

\geometry{margin=0.7in}

\setlength{\parskip}{0.5pt}
\setlist{nosep}  
\usepackage[compact]{titlesec}
\usepackage{setspace}
\setstretch{1.0}

\begin{document}

%------------------------------------------
% HEADER
%------------------------------------------
\begin{center}
    \huge \textbf{Yongkang Cheng}\\
    \vspace{0.2cm}
    \footnotesize
    \href{https://chengyongkang.me/}{\uline{chengyongkang.me}} \,\textbar{}\,
    437-663-2855 \,\textbar{}\,
    \href{https://www.github.com/Ken-2511/}{\uline{github.com/Ken-2511}} \,\textbar{}\,
    \href{mailto:iwmain@outlook.com}{\uline{iwmain@outlook.com}} \,\textbar{}\,
    \href{https://www.linkedin.com/in/chengyongkang/}{\uline{linkedin.com/in/chengyongkang}}
\end{center}

\vskip -2ex
\noindent\rule{\linewidth}{2pt}

%------------------------------------------
% EDUCATION
%------------------------------------------
\section*{\textbf{EDUCATION}}
\textbf{University of Toronto (St. George Campus), Toronto, ON} \hfill Sep 2023 - May 2028 (expected)\\
Bachelor of Applied Science in Computer Engineering + PEY Co-op (cGPA: 3.87/4.0) \\
Relevant Courses: Applied Fundamentals of Deep Learning, Software Design and Communication

\noindent\rule{\linewidth}{1pt}

%------------------------------------------
% TECHNICAL SKILLS
%------------------------------------------
% @ChatGPT: when you are modifying this section, you may want to remove some of the skills that are less relevant to the job you are applying for.
\section*{\textbf{TECHNICAL SKILLS}}
\begin{itemize}[leftmargin=0.2in]
    \item \textbf{Programming:} Python, C/C++, Node.js, Java, Verilog, Assembly (RISC-V)
    \item \textbf{AI \& Data:} PyTorch, LangChain, NumPy, Pandas, Matplotlib
    \item \textbf{Web \& Backend:} React, FastAPI, Node.js, Nginx, Docker, SQL/NoSQL
    \item \textbf{Hardware \& Tools:} Git, SSH, Arduino, Raspberry Pi, FPGA, LTSpice, Quartus, ModelSim, STM32
\end{itemize}

\noindent\rule{\linewidth}{1pt}

%------------------------------------------
% EXPERIENCE
%------------------------------------------
\section*{\textbf{EXPERIENCE}}

\noindent\href{https://github.com/Ken-2511/ppm-psk-visualize}
{\uline{\textbf{Research Assistant, Ultra-Wideband Receiver Design}}} \hfill Jun 2025 - Jul 2025\\
\textit{Research Intern, X-Lab, University of Toronto} \hfill \textit{Toronto, ON}
\begin{itemize}[leftmargin=0.2in]
    \item Verified hybrid PPM+PSK TX chip pre tape-out; built Python/Simulink pipelines for 2ns symbol sync and carrier recovery under discontinuous 4GHz.
	\item Built pulse-position detection and K-means cluster calibration to mitigate cross-modulation 100ps timing shifts.
    \item Presented at Undergraduate Engineering Research Day with an interactive hybrid-modulation demo site.
\end{itemize}

\vspace{0.3cm}
\noindent\textbf{Research Assistant, Wireless Power Transfer Coil Design} \hfill Jul 2025 - Aug 2025\\
\textit{Research Intern, X-Lab, University of Toronto} \hfill \textit{Toronto, ON}
\begin{itemize}[leftmargin=0.2in]
    \item Designed 13.56MHz WPT coils for a BCI implant (3mm×8mm RX, \textasciitilde20mm link).
    \item Ran HFSS sweeps (turns, trace size, TX diameter) to quantify impacts on coupling ($k$) and quality factor ($Q$).
    \item Produced PCB layouts with tuning plan; distilled design rules and prepared prototypes for validation.
\end{itemize}


\vspace{0.3cm}
\noindent\href{https://volun-track.web.app}{\uline{
\textbf{Frontend Manager, Voluntrack.org}}} \textit{React, Figma, MS Project} \hfill May 2024 - Present\\
\textit{Volunteer, Non-profit Organization Voluntrack.org} \hfill \textit{Remote}
\begin{itemize}[leftmargin=0.2in]
    \item Led a 4-person team to redesign the web interface using React.js, improving user engagement by 20\%.
    \item Designed UI in Figma and managed tasks with GitHub Project.
    \item Integrated Firebase for secure volunteer data storage and real-time search, allowing fuzzy search and filtering.
\end{itemize}

\noindent\rule{\linewidth}{1pt}

%------------------------------------------
% PROJECTS
%------------------------------------------
\section*{\textbf{PROJECTS}}

\noindent\href{https://github.com/Ken-2511/HandwritingRecognition}{\uline{
\textbf{Project Lead, Handwritten Text Recognition (University of Toronto)}}} \hfill Jun 2024 - Aug 2024
\begin{itemize}[leftmargin=0.2in]
    \item Led a remote team to develop a PyTorch-based CRNN model for handwritten text recognition.
    \item Achieved 87\% word-level and 95\% character-level accuracy on the test set with 10,000+ samples.
    \item Deployed connected-pixel algorithms for word positioning and word segmentation, processing 1024$\times$1024 images in less than 4 seconds.
\end{itemize}

\vspace{0.3cm}
\noindent\href{https://github.com/XiuShw/Wrong-Tree}{\uline{
\textbf{Core AI Developer, Wrong-Tree Unity Game}}} \hfill Dec 2024 - Jan 2025
\begin{itemize}[leftmargin=0.2in]
    \item Designed 9-state finite state machine (Idle, Wander, Share, Steal, Flee) for intelligent NPC behaviors in Unity.
    \item Implemented AI decision-making algorithms with proximity detection and dynamic reputation scoring for realistic social interactions.
    \item Created multi-NPC interaction framework supporting simultaneous share/steal operations with visual feedback animations.
    \item Built modular utility systems including random walk algorithms and distance-based targeting for scalable codebase.
\end{itemize}

\vspace{0.3cm}
\noindent\href{https://github.com/Ken-2511/wellness-room-expansion}{\uline{
\textbf{Project Manager, Wellness Room Expansion (University of Toronto)}}} \hfill Jan 2024 -- Apr 2024
\begin{itemize}[leftmargin=0.2in]
    \item Led a cross-functional team of 6 students to redesign and prototype a wellness room, endorsed by the client.
    % @ChatGPT: if needed, replace Microsoft Project with Gantt Chart
    \item Streamlined task management using Microsoft Project, tracking over 100+ tasks to completion on schedule.
    \item Incorporating client feedback, generated 100+ initial ideas and finalized 3 proposed designs.
    \item Designed and visualized proposals using Rhino and Blender 3D models, reducing client feedback iterations and improving decision-making efficiency.
    \item Conducted research on noise isolation, light intensity, and light temperature to deliver optimal design solutions.
\end{itemize}

\vspace{0.3cm}
\noindent\href{https://github.com/Ken-2511/utek_disaster_communication_system}{\uline{
\textbf{UTEK Wildfire Disaster Communication System}}} \hfill Jan 2025
\begin{itemize}[leftmargin=0.2in]
    \item \textbf{Top 8 Finalist} in University of Toronto Engineering Competition (UTEK) among all competing teams.
    \item Built real-time communication platform with Python, Streamlit, and Flask for wildfire emergency response coordination.
    \item Implemented interactive map interface using Folium for visualizing fire incidents and location-based alerts.
    \item Developed bidirectional communication system between residents and rescue teams with photo upload capabilities.
    \item Created severity classification system with automated risk assessment for intelligent resource allocation.
    \item Designed secure authentication portal for authorized rescue personnel with role-based access control.
\end{itemize}

\vspace{0.3cm}
\noindent\href{https://github.com/Ken-2511/walrus_test}{\uline{
\textbf{Fourier Epicycle Drawing Visualization System}}} \hfill May 2025
\begin{itemize}[leftmargin=0.2in]
    \item Built an interactive Python/Pygame app to draw strokes and visualize their Fourier decomposition as animated epicycles in just 3 hours using the MCP (Model Context Protocol) technique.
    \item Implemented stroke preprocessing (merging, equidistant resampling) for accurate analysis.
    \item Animated rotating circles (epicycles) to demonstrate Fourier series reconstruction of closed curves.
    \item Developed modular utilities for stroke processing, Fourier transform, and Matplotlib-based plots.
    \item Designed extensible architecture for saving/loading, real-time adjustment, and multi-color support.
    \item Produced educational outputs and visualizations for teaching and presentations.
\end{itemize}

\vspace{0.3cm}
\noindent\href{https://github.com/alexzjm/ece243-sound-synthesizer}{\uline{
\textbf{FPGA Polyphonic Synthesizer (DE1‑SoC)}}} \hfill Mar 2025
\begin{itemize}[leftmargin=0.2in]
    \item Implemented a 20‑voice digital synthesizer in C for a Nios‑V soft‑core, streaming \textbf{8 kHz Q15 audio} through the on‑chip Audio FIFO.  
    \item Replaced all floating‑point math with \textbf{32‑bit phase accumulators} and fixed‑point kernels for sine, square, triangle, and sawtooth waves, enabling real‑time mixing and envelope processing.  
    \item Designed an \textbf{ADSR envelope engine} driven by slide‑switch “knobs” and pushbuttons; state changes are visualized on a double‑buffered 320 × 240 VGA UI.  
    \item Integrated PS/2 keyboard interrupts for sub‑µs latency note‑on/off events; on‑screen piano keys light up in sync with hardware playback.  
    \item Built modular drawing primitives (Bresenham, bitmap blits) to render live waveforms and icons; architecture supports future effects or MIDI input with minimal refactor.  
\end{itemize}


\vspace{0.3cm}
\noindent\textbf{City Mapify – Interactive City Mapping Application (University of Toronto)} \hfill Jan 2025 - Apr 2025
\begin{itemize}[leftmargin=0.2in]
    \item Developed a high-performance mapping engine in C++ to process OpenStreetMap data and render city maps.
    \item Designed efficient spatial data structures (\textbf{quadtrees}) for dynamic querying and smooth zoom-based rendering.
    \item Implemented advanced pathfinding algorithms (\textbf{Dijkstra, A*, Simulated Annealing, Ant Colony Optimization}) for route planning and delivery optimization.
    \item Integrated real-time features like day/night mode, weather data, and AI-powered route descriptions.
    \item Enhanced performance with multithreading (OpenMP) and RESTful API integration (libcurl).
\end{itemize}


\vspace{0.3cm}
\noindent\href{https://github.com/Ken-2511/Diary-with-ChatGPT-Comment}{\uline{
\textbf{Diary with AI Feedback}}} \hfill Sep 2023 -- On Going
\begin{itemize}[leftmargin=0.2in]
    \item Designed and implemented a journaling program integrated with OpenAI’s GPT API, generating insightful feedback and suggestions for over 750 diary entries.
    \item Developed a diary sorting algorithm to retrieve contextually similar past entries by vector search, enhancing user experience and maintaining API costs below 0.2\$ per call.
    \item Optimized data-sorting pipelines and API request processes, reducing average diary load time from 10s to 0.5s, enabling seamless daily use.
\end{itemize}


\vspace{0.3cm}
\noindent\href{https://github.com/Ken-2511/ECE241-Project}{\uline{
\textbf{Verilog Pac-Man Game (University of Toronto)}}} \hfill Nov 2024
\begin{itemize}[leftmargin=0.2in]
    \item Created a Pac-Man-style FPGA game using Verilog supporting PS/2 keyboard input and VGA output.
    \item Debugged signal synchronization issues and state-machine logic, boosting overall stability and playability.
    \item Automated image conversion using Python + OpenCV for seamless integration of game graphics.
    \item Prototypeed the game using Pygame, ensuring accurate emulation of the FPGA version for agile development.
\end{itemize}

\vspace{0.3cm}
\noindent\href{https://github.com/Ken-2511/WillPower}{\uline{
\textbf{WillPower \textbar{} Time Management \& Monitoring}}} \hfill Jan 2025 - Present
\begin{itemize}[leftmargin=0.2in]
    \item Built a modular system with Raspberry Pi capturing images and sending them to a Windows host for local storage and analysis.
    \item Deployed Nginx, FastAPI, and libcurl for data transfer, facilitating real-time user monitoring and minimal downtime.
    % @ChatGPT: for the bullet point below you can change the technologies used to match the job you are applying for.
    \item Currently exploring Azure Face APIs and transfer learning for user-behavior analysis on a dataset of over 180,000 images.
\end{itemize}

\vspace{0.3cm}
\noindent\href{https://chengyongkang.me/chat}{\uline{
\textbf{Self-Clone Chatbot with Diary Database}}} \hfill Oct 2024 - Present
% @ChatGPT: for this project, if the job doesn't have so much related technical skills, you can remove this project.
\begin{itemize}[leftmargin=0.2in]
    \item Built a self-hosted AI-powered chatbot that replicates personal interaction styles, deployed using React.js, FastAPI, and Nginx on a Raspberry Pi.
    \item Integrated OpenAI API and a NoSQL database for real-time Q\&A functionality with personal diary data.
    \item Ensured secure and seamless remote access by implementing TLS encryption, DDNS, and optimizing for daily traffic from personal networks.
\end{itemize}

\vspace{0.3cm}
\noindent\href{https://github.com/Ken-2511/Photogate}{\uline{
\textbf{Photogate Speed Measurement System}}} \hfill High School Project
\begin{itemize}[leftmargin=0.2in]
    \item Designed and built 10 laser-based speed measurement units with 7-segment displays for high school physics education.
    \item Achieved sub-150us measurement precision using Arduino microcontrollers and custom PCB design.
    \item Developed custom infrared communication protocol supporting 32-byte data transmission for wireless control.
    \item Created Python GUI using Tkinter for experiment control, real-time monitoring, and data export capabilities.
    \item Integrated 3D printed components and aluminum framework for durable classroom-ready construction.
\end{itemize}

\vspace{0.3cm}
\noindent\href{https://github.com/Ken-2511/AcornCarwler}{\uline{
\textbf{Acorn Course Timetable Monitor Crawler}}} \hfill Personal Project
\begin{itemize}[leftmargin=0.2in]
    \item Built automated Python web crawler to monitor University of Toronto's Acorn course timetable system for real-time changes.
    \item Implemented intelligent change detection algorithms with robust error handling and automatic recovery mechanisms.
    \item Developed notification system for course availability updates, helping students secure spots in full courses.
    \item Created comprehensive logging system for tracking enrollment patterns and course schedule modifications.
\end{itemize}

\noindent\rule{\linewidth}{1pt}

%------------------------------------------
% AWARDS & ACCOMPLISHMENTS
%------------------------------------------
\section*{\textbf{AWARDS \& ACCOMPLISHMENTS}}

\noindent\textbf{University of Toronto Excellence Award (UTEA)} \hfill Apr 2025
\begin{itemize}[leftmargin=0.2in]
    \item Awarded UTEA for top academic performance and research potential.
    \item Completed a 14-week full-time research project with faculty supervision.
    \item Received \$7,500 scholarship for research excellence and inclusion.
\end{itemize}

\vspace{0.3cm}
\noindent\textbf{ECE Awards \& Dean's List Scholar (UofT)} \hfill Sep 2024
\begin{itemize}[leftmargin=0.2in]
    \item Recognized for outstanding academic performance.
\end{itemize}

% \vspace{0.3cm}
% \noindent\textbf{American Computer Science League (ACSL) - Bronze Prize} \hfill Jan 2021
% \begin{itemize}[leftmargin=0.2in]
%     \item Placed in top 10\% overall, with top-20\% scores in the 4th round, after 60 hours of training.
% \end{itemize}

\end{document}
