\documentclass[letterpaper,10pt]{article}
\usepackage{geometry}
\geometry{margin=0.65in}
\usepackage{enumitem}
\usepackage[hidelinks]{hyperref}
\usepackage[normalem]{ulem}
\pagenumbering{gobble}

\setlength{\parskip}{0.5pt}
\setlist{nosep,leftmargin=0.18in}
\usepackage[compact]{titlesec}
\titlespacing*{\section}{0pt}{4pt}{2pt}
\usepackage{setspace}
\setstretch{1.0}

\begin{document}

%------------------------------------------
% HEADER
%------------------------------------------
\begin{center}
    {\huge \textbf{Yongkang Cheng}}\\[2pt]
    {\footnotesize
    \href{https://chengyongkang.me/}{\uline{chengyongkang.me}} \;\textbar\; 
    437-663-2855 \;\textbar\; 
    \href{https://www.github.com/Ken-2511/}{\uline{github.com/Ken-2511}} \;\textbar\; 
    \href{mailto:iwmain@outlook.com}{\uline{iwmain@outlook.com}} \;\textbar\; 
    \href{https://www.linkedin.com/in/chengyongkang/}{\uline{linkedin.com/in/chengyongkang}}}
\end{center}
\vskip -1.5ex
\noindent\rule{\linewidth}{1.6pt}

%------------------------------------------
% EDUCATION
%------------------------------------------
\section*{\textbf{EDUCATION}}
\textbf{University of Toronto (St. George Campus), Toronto, ON} \hfill Sep 2023 - May 2028 (expected)\\
Bachelor of Applied Science in Computer Engineering + PEY Co-op (cGPA: 3.87/4.0) \\
Relevant Courses: Applied Fundamentals of Deep Learning, Software Design and Communication

\noindent\rule{\linewidth}{1pt}

%------------------------------------------
% TECHNICAL SKILLS
%------------------------------------------
% @ChatGPT: when you are modifying this section, you may want to remove some of the skills that are less relevant to the job you are applying for.
\section*{\textbf{TECHNICAL SKILLS}}
\begin{itemize}[leftmargin=0.2in]
    \item \textbf{Programming:} Python, C/C++, Node.js, Java, Verilog, Assembly (RISC-V)
    \item \textbf{AI \& Data:} PyTorch, LangChain, NumPy, Pandas, Matplotlib
    \item \textbf{Web \& Backend:} React, FastAPI, Node.js, Nginx, Docker, SQL/NoSQL
    \item \textbf{Hardware \& Tools:} Git, SSH, Arduino, Raspberry Pi, FPGA, LTSpice, Quartus, ModelSim, STM32
\end{itemize}

\noindent\rule{\linewidth}{1pt}

%------------------------------------------
% EXPERIENCE
%------------------------------------------
\section*{\textbf{EXPERIENCE}}

\noindent\href{https://github.com/Ken-2511/ppm-psk-visualize}
{\uline{\textbf{Research Assistant, Ultra-Wideband Receiver Design}}} \hfill May 2025 - Jul 2025\\
\textit{Research Intern, X-Lab, University of Toronto} \hfill \textit{Toronto, ON}
\begin{itemize}[leftmargin=0.2in]
    \item Verified a hybrid 2-PPM + 8-PSK TX chip pre-tapeout; built Python/Simulink pipelines for 2 ns symbol sync and carrier recovery under discontinuous 4 GHz.
    \item Built pulse-position detection and K-means timing/phase calibration for constellation stabilization and PPM demodulation.
    \item Achieved zero-BER demodulation across 2,500 symbols under $\geq 13\,\mathrm{dB}$ SNR (AWGN) and $\pi/16$ phase jitter.
    \item Presented at Undergraduate Engineering Research Day with a \href{https://docs.google.com/presentation/d/1h4lmc_HQLzNvtGE4oE1jedOlNVZjL3iztEyGKnC-ico/edit?usp=sharing}{\uline{poster}} and an interactive demo \href{https://github.com/Ken-2511/ppm-psk-visualize}{\uline{site}}.
\end{itemize}

\vspace{0.3cm}
\noindent\textbf{Research Assistant, Wireless Power Transfer Coil Design} \hfill Jul 2025 - Aug 2025\\
\textit{Research Intern, X-Lab, University of Toronto} \hfill \textit{Toronto, ON}
\begin{itemize}[leftmargin=0.2in]
    \item Designed a resonant 13.56 MHz wireless power transfer coil pair enabling reliable power delivery to a brain-computer interface implant across a 20 mm separation.
    \item Ran HFSS sweeps (turns, trace width, TX diameter) to optimize coupling ($k > 0.01$) and quality factor ($Q > 28$).
    \item Performed electromagnetic field simulations and resonant frequency tuning to improve link efficiency under load.
    \item Designed PCB layouts incorporating capacitive tuning networks; fabricated and tested 3 TX and 11 RX prototype boards to validate coupling efficiency.
\end{itemize}


\vspace{0.3cm}
\noindent\href{https://volun-track.web.app}{\uline{
\textbf{Frontend Manager, Voluntrack.org}}} \textit{React, Figma, MS Project} \hfill May 2024 - Present\\
\textit{Volunteer, Non-profit Organization Voluntrack.org} \hfill \textit{Remote}
\begin{itemize}[leftmargin=0.2in]
    \item Led a 4-person team to redesign the web interface using React.js, improving user engagement by 20\%.
    \item Designed UI in Figma and managed tasks with GitHub Project.
    \item Integrated Firebase for secure volunteer data storage and real-time search, allowing fuzzy search and filtering.
\end{itemize}

\noindent\rule{\linewidth}{1pt}

%------------------------------------------
% PROJECTS
%------------------------------------------
\section*{\textbf{PROJECTS}}

\noindent\href{https://github.com/Ken-2511/HandwritingRecognition}{\uline{
\textbf{Project Lead, Handwritten Text Recognition}}} \hfill Jun 2024 - Aug 2024\\
\textit{Course Project, University of Toronto}
\begin{itemize}[leftmargin=0.2in]
    \item Led a remote team to develop a PyTorch-based CRNN model for handwritten text recognition.
    \item Achieved 87\% word-level and 95\% character-level accuracy on the test set with 10,000+ samples.
    \item Deployed connected-pixel algorithms for word positioning and word segmentation, processing 1024$\times$1024 images in less than 4 seconds.
\end{itemize}

\vspace{0.3cm}
\noindent\href{https://github.com/XiuShw/Wrong-Tree}{\uline{
\textbf{Core AI Developer, Wrong-Tree Unity Game}}} \hfill May 2025
\begin{itemize}[leftmargin=0.2in]
    \item Designed 9-state finite state machine (Idle, Wander, Share, Steal, Flee) for intelligent NPC behaviors in Unity.
    \item Implemented AI decision-making algorithms with proximity detection and dynamic reputation scoring for realistic social interactions.
    \item Created multi-NPC interaction framework supporting simultaneous share/steal operations with visual feedback animations.
    \item Built modular utility systems including random walk algorithms and distance-based targeting for scalable codebase.
\end{itemize}

\vspace{0.3cm}
\noindent\href{https://github.com/Ken-2511/wellness-room-expansion}{\uline{
\textbf{Project Manager, Wellness Room Expansion}}} \hfill Jan 2024 - Apr 2024\\
\textit{Course Project, University of Toronto}
\begin{itemize}[leftmargin=0.2in]
    \item Led a cross-functional team of 6 students to redesign and prototype a wellness room, endorsed by the client.
    \item Streamlined task management using Microsoft Project, tracking over 100+ tasks to completion on schedule.
    \item Incorporating client feedback, generated 100+ initial ideas and finalized 3 proposed designs.
    \item Designed and visualized proposals using Rhino and Blender 3D models, reducing client feedback iterations and improving decision-making efficiency.
    \item Conducted research on noise isolation, light intensity, and light temperature to deliver optimal design solutions.
\end{itemize}

\vspace{0.3cm}
\noindent\href{https://github.com/Ken-2511/utek_disaster_communication_system}{\uline{
\textbf{Wildfire Disaster Communication System}}} \hfill Jan 2025\\
\textit{Backend Developer, University of Toronto Engineering Competition (UTEK)}
\begin{itemize}[leftmargin=0.2in]
    \item \textbf{Top 8 Finalist} in University of Toronto Engineering Competition among all competing teams.
    \item Built real-time communication platform with Python, Streamlit, and Flask for wildfire emergency response coordination.
    \item Implemented interactive map interface using Folium for visualizing fire incidents and location-based alerts.
    \item Developed bidirectional communication system between residents and rescue teams with photo upload capabilities.
    \item Created severity classification system with automated risk assessment for intelligent resource allocation.
    \item Designed secure authentication portal for authorized rescue personnel with role-based access control.
\end{itemize}

\vspace{0.3cm}
\noindent\href{https://github.com/alexzjm/ece243-sound-synthesizer}{\uline{
\textbf{FPGA Polyphonic Synthesizer (DE1‑SoC)}}} \hfill Mar 2025\\
\textit{Course Project, University of Toronto}
\begin{itemize}[leftmargin=0.2in]
    \item Implemented a 20‑voice digital synthesizer in C for a Nios‑V soft‑core, streaming \textbf{8 kHz Q15 audio} through the on‑chip Audio FIFO.  
    \item Replaced all floating‑point math with \textbf{32‑bit phase accumulators} and fixed‑point kernels for sine, square, triangle, and sawtooth waves, enabling real‑time mixing and envelope processing.  
    \item Designed an \textbf{ADSR envelope engine} driven by slide‑switch “knobs” and pushbuttons; state changes are visualized on a double‑buffered 320 × 240 VGA UI.  
    \item Integrated PS/2 keyboard interrupts for sub‑µs latency note‑on/off events; on‑screen piano keys light up in sync with hardware playback.  
    \item Built modular drawing primitives (Bresenham, bitmap blits) to render live waveforms and icons; architecture supports future effects or MIDI input with minimal refactor.  
\end{itemize}


\vspace{0.3cm}
\noindent\textbf{City Mapify – Interactive City Mapping Application (University of Toronto)} \hfill Jan 2025 - Apr 2025\\
\textit{Course Project, University of Toronto}
\begin{itemize}[leftmargin=0.2in]
    \item Built a C++ mapping engine parsing \textbf{2GB} OpenStreetMap data with QuadTree indexing at 60 FPS.
    \item Implemented A*/Dijkstra pathfinding and delivery optimization (Simulated Annealing, ACO) for \textbf{250+} packages.
\end{itemize}


\vspace{0.3cm}
\noindent\href{https://github.com/Ken-2511/Diary-with-ChatGPT-Comment}{\uline{
\textbf{Diary with AI Feedback}}} \hfill Sep 2023 -- On Going\\
\textit{Personal Project, React.js, FastAPI, OpenAI API}
\begin{itemize}[leftmargin=0.2in]
    \item Designed and implemented a journaling program integrated with OpenAI’s GPT API, generating insightful feedback and suggestions for over 750 diary entries.
    \item Developed a diary sorting algorithm to retrieve contextually similar past entries by vector search, enhancing user experience and maintaining API costs below 0.2\$ per call.
    \item Optimized data-sorting pipelines and API request processes, reducing average diary load time from 10s to 0.5s, enabling seamless daily use.
\end{itemize}


\vspace{0.3cm}
\noindent\href{https://github.com/Ken-2511/ECE241-Project}{\uline{
\textbf{Verilog Pac-Man Game (DE1-SoC)}}} \hfill Nov 2024\\
\textit{Course Project, University of Toronto}
\begin{itemize}[leftmargin=0.2in]
    \item Created a Pac-Man-style FPGA game using Verilog supporting PS/2 keyboard input and VGA output.
    \item Debugged signal synchronization issues and state-machine logic, boosting overall stability and playability.
    \item Automated image conversion using Python + OpenCV for seamless integration of game graphics.
    \item Prototyped the game using Pygame, ensuring accurate emulation of the FPGA version for agile development.
\end{itemize}

\vspace{0.3cm}
\noindent\href{https://github.com/Ken-2511/WillPower}{\uline{
\textbf{WillPower \textbar{} Time Management \& Monitoring}}} \hfill Jan 2025 - Present\\
\textit{Personal Project, Raspberry Pi, FastAPI, Azure Face API}
\begin{itemize}[leftmargin=0.2in]
    \item Built a modular system with Raspberry Pi capturing images and sending them to a Windows host for local storage and analysis.
    \item Deployed Nginx, FastAPI, and libcurl for data transfer, facilitating real-time user monitoring and minimal downtime.
    \item Currently exploring Azure Face APIs and transfer learning for user-behavior analysis on a dataset of over 180,000 images.
\end{itemize}

\vspace{0.3cm}
\noindent\href{https://chengyongkang.me/chat}{\uline{
\textbf{Self-Clone Chatbot with Diary Database}}} \hfill Oct 2024 - Present\\
\textit{Personal Project, React.js, FastAPI, OpenAI API}
\begin{itemize}[leftmargin=0.2in]
    \item Built a self-hosted AI-powered chatbot that replicates personal interaction styles, deployed using React.js, FastAPI, and Nginx on a Raspberry Pi.
    \item Integrated OpenAI API and a NoSQL database for real-time Q\&A functionality with personal diary data.
    \item Ensured secure and seamless remote access by implementing TLS encryption, DDNS, and optimizing for daily traffic from personal networks.
\end{itemize}

\vspace{0.2cm}
\noindent\href{https://github.com/Ken-2511/Photogate}{\uline{
    \textbf{Photogate Speed Measurement System}}} \hfill High School Project\\
\textit{Personal Project, Arduino, Python, Tkinter}
\begin{itemize}[leftmargin=0.2in]
    \item Designed and built 10 laser-based speed measurement units with 7-segment displays for physics education.
    \item Achieved sub-150us measurement precision using Arduino microcontrollers and custom PCB design.
    \item Developed custom infrared communication protocol supporting 32-byte data transmission for wireless control.
    \item Implemented precise timing circuits with crystal oscillators and interrupt-driven measurement algorithms.
    \item Integrated 3D printed components and aluminum framework for durable classroom-ready construction.
\end{itemize}

\vspace{0.3cm}
\noindent\href{https://github.com/Ken-2511/AcornCarwler}{\uline{
\textbf{Acorn Course Timetable Monitor Crawler}}} \hfill Aug 2025\\
\textit{Personal Project, Python, BeautifulSoup, SQLite}
\begin{itemize}[leftmargin=0.2in]
    \item Built automated Python web crawler to monitor University of Toronto's Acorn course timetable system for real-time changes.
    \item Implemented intelligent change detection algorithms with robust error handling and automatic recovery mechanisms.
    \item Developed notification system for course availability updates, helping students secure spots in full courses.
    \item Created comprehensive logging system for tracking enrollment patterns and course schedule modifications.
\end{itemize}

\noindent\rule{\linewidth}{1pt}

%------------------------------------------
% AWARDS & ACCOMPLISHMENTS
%------------------------------------------
\section*{\textbf{AWARDS}}

\begin{itemize}[leftmargin=0.2in]
	\item \textbf{University of Toronto Excellence Award (UTEA)} \hfill Apr 2025\\
    \$7,500 scholarship for 6 students among 2nd to 4th ECE for research potential.
	\item \textbf{ECE Awards} \hfill Sep 2024\\
	Awarded to top 30 students in the first-year ECE program out of 300+ students.
	\item \textbf{Dean's List Scholar} \hfill Sep 2023 - Apr 2025\\
    Awarded for top academic standing (average \textgreater 80\%) for 4 consecutive semesters.
\end{itemize}

\end{document}
