\documentclass[11pt]{article}
\usepackage[margin=1in]{geometry}
\usepackage[hidelinks]{hyperref}
\usepackage{setspace}
\setstretch{1.05}
\pagenumbering{gobble}
\usepackage{parskip}
\setlength{\parindent}{0pt}
\usepackage[T1]{fontenc}
\usepackage{lmodern}

\begin{document}

{\Large \textbf{Yongkang Cheng}}\\[2pt]
\href{https://chengyongkang.me/}{chengyongkang.me} \;|\; 437-663-2855 \;|\; \href{mailto:iwmain@outlook.com}{iwmain@outlook.com} \;|\; \href{https://github.com/Ken-2511}{github.com/Ken-2511}

\today\\[0.8em]

Dear Tenstorrent Hiring Team,

I am a Computer Engineering PEY student at the University of Toronto (cGPA 3.87/4.0) and I am applying for the PCB Intern openings on the Power Electronics, High-Speed Design, and Digital Electronics tracks. Tenstorrent’s focus on building world-class boards that unlock custom RISC-V AI silicon matches the work I enjoy most: translating schematics into manufacturable layouts, validating them in the lab, and iterating with data. I am Toronto-based and available on-site.

My most recent X-Lab project centered on wireless power delivery for a brain–computer interface. I designed resonant 13.56 MHz coil pairs, modeled coupling in Ansys HFSS, and captured schematics/layout for tuning networks. After fabricating three transmitter and eleven receiver PCBs, I brought them up using VNAs, oscilloscopes, and adjustable loads to verify quality factor and efficiency. That cycle—simulation, layout, build, measurement, and quick rework—reinforced the disciplined process Tenstorrent follows across the PCB lifecycle.

Earlier, on our ultra-wideband receiver effort, I owned parts of the analog/digital verification flow. I scripted Python analyses to sweep symbol timing and jitter, correlated them with oscilloscope captures, and used lab instrumentation to validate that our hybrid 2-PPM + 8-PSK link stayed error-free across 2,500 symbols at $\geq 13\,\mathrm{dB}$ SNR. Working in a two-person team meant jumping between schematic tweaks, fixture builds, and measurement automation—habits that translate to supporting high-speed interconnect bring-up.

I pair that lab mindset with digital board experience. On FPGA projects such as the polyphonic synthesizer and Pac-Man game, I maintained clean Verilog/RTL, captured support circuitry, and built Python + ModelSim scripts to automate regressions. Those efforts made me comfortable managing board-level test plans, soldering fixes, and collaborating with classmates on layout decisions in tools like Altium Designer.

Across all of these projects I default to instrumenting prototypes, logging results, and iterating quickly—exactly what your PCB team needs when a board first powers on. I am excited to contribute to Tenstorrent by drafting schematics, tuning stackups, running HFSS or LTSpice analyses, and partnering with cross-functional teams to keep bring-up on schedule.

Thank you for your consideration. I would welcome the opportunity to discuss how I can support Tenstorrent’s PCB design and validation efforts this PEY term.

\vspace{1em}
Sincerely,\\[1.2em]
Yongkang Cheng

\end{document}
