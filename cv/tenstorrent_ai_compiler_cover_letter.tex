\documentclass[11pt]{article}
\usepackage[margin=1in]{geometry}
\usepackage[hidelinks]{hyperref}
\usepackage{setspace}
\setstretch{1.05}
\pagenumbering{gobble}
\usepackage{parskip}
\setlength{\parindent}{0pt}
\usepackage[T1]{fontenc}
\usepackage{lmodern}

\begin{document}

{\Large \textbf{Yongkang Cheng}}\\[2pt]
\href{https://chengyongkang.me/}{chengyongkang.me} \;|\; 437-663-2855 \;|\; \href{mailto:iwmain@outlook.com}{iwmain@outlook.com} \;|\; \href{https://github.com/Ken-2511}{github.com/Ken-2511}

\today

\vspace{0.6em}
Dear Hiring Manager,

I am excited to apply for the Low-Level Software Intern role at Tenstorrent. As a BASc Computer Engineering student at the University of Toronto focused on systems, performance, and close-to-metal development, I am drawn to Tenstorrent’s vision of unifying hardware and software to power next-generation AI. I am ready to contribute onsite in Toronto, working with engineers who push kernel-level performance and ML workloads.

Beyond what is in my resume, several projects from my main resume show I learn fast and ship:
\begin{itemize}
    \item \textbf{\href{https://github.com/Ken-2511/HandwritingRecognition}{\underline{Handwritten Text Recognition:}}} Led a PyTorch CRNN effort with strong word/character accuracy, giving me intuition for how low-level kernels surface at the framework level.
    \item \textbf{\href{https://github.com/Ken-2511/Diary-with-ChatGPT-Comment}{\underline{Diary with AI Feedback:}}} Optimized end-to-end data and request pipelines, reducing average load time from 10 s to 0.5 s through measurement-driven iteration.
    \item \textbf{\href{https://github.com/Ken-2511/PhotogateSpeedMeasurement}{\underline{Photogate Speed Measurement:}}} Built laser-based timing with sub-150 $\mu$s precision on microcontrollers, reinforcing comfort with interrupts, timers, and debouncing.
    \item \textbf{\href{https://github.com/Ken-2511/WillPower}{\underline{WillPower:}}} Deployed a Raspberry Pi–based data path with Nginx, FastAPI, and libcurl, strengthening my systems wiring and reliability instincts.
\end{itemize}

I am comfortable diving into unfamiliar codebases, instrumenting and profiling to find bottlenecks, and collaborating to get real workloads running. I bring strong C/C++, RISC-V familiarity, a bias for fixed-point and vectorizable kernels when appropriate, and a disciplined approach to debugging. Most of all, I learn extremely fast and enjoy turning performance mysteries into clear, testable hypotheses.

Thank you for your time and consideration. I would welcome the chance to discuss how I can contribute to Tenstorrent’s low-level software stack and help accelerate real ML workloads on your hardware.

Sincerely,\\[1.0em]
Yongkang Cheng

\end{document}

