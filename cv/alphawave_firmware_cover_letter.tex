\documentclass[11pt]{article}
\usepackage[margin=1in]{geometry}
\usepackage[hidelinks]{hyperref}
\usepackage{setspace}
\setstretch{1.05}
\pagenumbering{gobble}
\usepackage{parskip}
\setlength{\parindent}{0pt}
\usepackage[T1]{fontenc}
\usepackage[utf8]{inputenc} % Ensure UTF-8 characters (≥, non-breaking dashes, etc.) compile under pdfLaTeX
\usepackage{lmodern}

\begin{document}

{\Large \textbf{Yongkang Cheng}}\\[2pt]
\href{https://chengyongkang.me/}{chengyongkang.me} \;|\; 437-663-2855 \;|\; \href{mailto:iwmain@outlook.com}{iwmain@outlook.com} \;|\; \href{https://github.com/Ken-2511}{github.com/Ken-2511}

\today\\[0.8em]

Dear Hiring Team,

I’m applying for the Alphawave Firmware Intern role because I like the exact moment when a line of code becomes a timing edge or a measurable signal—and I want more of that. I’ve gone deep in software (C++ performance work, Python/ML projects), but real silicon / firmware chances have been fewer; I want to bring strong software habits while growing fast on the hardware side.

Briefly: I’ve built enough FPGA / embedded pieces (PS/2 path, VGA timing, fixed‑point audio on a Nios‑V RISC‑V soft core) to learn to think in clocks, resource trade‑offs, and clean reset/state contracts. I also self‑studied basic wireless/digital comms while building a hybrid 2‑PPM + 8‑PSK demod pipeline that I calibrated to zero‑BER at $\geq 13$ dB SNR—giving me a taste for adaptation and link‑style tuning.

I like turning manual steps into tiny scripts (Makefile builds, vector generators, quick log diffs). Day one I can write clear Verilog/SystemVerilog control logic, add lightweight counters/status, and build Python/C harnesses to exercise registers and capture metrics—while actively leveling up in SerDes concepts, assertions/coverage, and structured lab bring‑up.

Your focus on high‑performance connectivity and reusable internal kits matches exactly how I like to work: make behavior observable, automate the boring parts, keep code readable. I’d love to contribute useful tooling and clean firmware while strengthening my hardware intuition.

Thank you for your consideration.

\vspace{1em}
Sincerely,\\[1.2em]
Yongkang Cheng

\end{document}
